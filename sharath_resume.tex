%----------------------------------------------------------------------------------------
%	DOCUMENT DEFINITION
%----------------------------------------------------------------------------------------

% article class because we want to fully customize the page and not use a cv template
\documentclass[a4paper,10pt]{article}

%----------------------------------------------------------------------------------------
%	FONT
%----------------------------------------------------------------------------------------

% % fontspec allows you to use TTF/OTF fonts directly
% \usepackage{fontspec}
% \defaultfontfeatures{Ligatures=TeX}

% % modified for ShareLaTeX use
% \setmainfont[
% SmallCapsFont = Fontin-SmallCaps.otf,
% BoldFont = Fontin-Bold.otf,
% ItalicFont = Fontin-Italic.otf
% ]
% {Fontin.otf}

%----------------------------------------------------------------------------------------
%	PACKAGES
%----------------------------------------------------------------------------------------
\usepackage{url}
\usepackage{parskip} 	

%other packages for formatting
\RequirePackage{color}
\RequirePackage{graphicx}
\usepackage[usenames,dvipsnames]{xcolor}
\usepackage[scale=0.9, bottom=0mm, top =0.75cm]{geometry}
% \usepackage[scale=0.9, bottom=0mm, top =1.2cm, showframe]{geometry}

%tabularx environment
\usepackage{tabularx}

%for lists within experience section
\usepackage{enumitem}

% centered version of 'X' col. type
\newcolumntype{C}{>{\centering\arraybackslash}X} 

%to prevent spillover of tabular into next pages
\usepackage{supertabular}
\usepackage{tabularx}
\newlength{\fullcollw}
\setlength{\fullcollw}{0.47\textwidth}

%custom \section
\usepackage{titlesec}				
\usepackage{multicol}
\usepackage{multirow}

%CV Sections inspired by: 
%http://stefano.italians.nl/archives/26
\titleformat{\section}{\Large\raggedright}{}{0em}{}[\titlerule]
\titlespacing{\section}{0pt}{10pt}{10pt}

%for publicationshttps://www.overleaf.com/project/6390f017c5f74f795fdfe0e5
% \usepackage[style=authoryear,sorting=ynt, maxbibnames=2]{biblatex}

%Setup hyperref package, and colours for links
\usepackage[unicode, draft=false]{hyperref}
\definecolor{linkcolour}{rgb}{0,0,0}  %{0,0.2,0.6} for blue
\hypersetup{colorlinks,breaklinks,urlcolor=linkcolour,linkcolor=linkcolour}
% \addbibresource{citations.bib}
% \setlength\bibitemsep{1em}

%for social icons
\usepackage{fontawesome5}
\usepackage{ifthen}


%debug page outer frames
%\usepackage{showframe}

%----------------------------------------------------------------------------------------
%	BEGIN DOCUMENT
%----------------------------------------------------------------------------------------
\begin{document}

% non-numbered pages
\pagestyle{empty} 
%----------------------------------------------------------------------------------------
%	TITLE
%----------------------------------------------------------------------------------------

% \begin{tabularx}{\linewidth}{ @{}X X@{} }
% \huge{Your Name}\vspace{2pt} & \hfill \emoji{incoming-envelope} email@email.com \\
% \raisebox{-0.05\height}\faGithub\ username \ | 
% \raisebox{-0.00\height}\faLinkedin\ username \ | \ \raisebox{-0.05\height}\faGlobe \ mysite.com  & \hfill \emoji{calling} number
% \end{tabularx}

\begin{tabularx}{\linewidth}{@{} C @{}}
    \Huge\textrm{Sharath Hebbur Shivakumar} \\[7.5pt]
        \href{https://github.com/sharathat45}{\raisebox{-0.05\height}\faGithub\ sharathat45} \ $|$ \ 
        \href{https://www.linkedin.com/in/sharath-hs/}{\raisebox{-0.05\height}\faLinkedin\ Sharath Hebbur Shivakumar} \ $|$ \ 
        \href{mailto:shebburs@purdue.edu}{\raisebox{-0.05\height}\faEnvelope \ shebburs@purdue.edu} \ $|$ \ 
        \href{tel:+17657673705}{\raisebox{-0.05\height}\faMobile \ +1 765-767-3705} \\
\end{tabularx}

%----------------------------------------------------------------------------------------
% OBJECTIVE
%----------------------------------------------------------------------------------------
%Interests/ Keywords/ Summary
% \section{Objective}
%     Seeking Full time opportunities (from May/July 2024).

%----------------------------------------------------------------------------------------
%	EDUCATION
%----------------------------------------------------------------------------------------
\vspace{-0.1cm}
\section{Education}

\begin{tabularx}{\linewidth}{ @{}l X@{} }
    \textbf{Master's in Electrical and Computer Engineering at Purdue University}, USA & \hfill {\small Aug 2022 - May 2024} \\[2.75pt]
    % Specialization in VLSI and Circuit Design &  \hfill GPA: 3.77/4.0\\
    {\small \textbf{Courses}: VLSI Design, SoC Design, GPU Arch, Advanced Computer Architecture, Parallel Systems} &  \hfill {\small GPA: 3.7/4.0}\\
\end{tabularx} 

\begin{tabularx}{\linewidth}{ @{}l X@{} }
    \textbf{Bachelor's Degree at PES University}, India & \hfill {\small Aug 2016 - Sept 2020} \\[2.75pt]
    {\small Electrical and Electronics Engineering, Graduated with Medal, Honors and Rank} & \hfill {\small GPA: 9.07/10.0}
\end{tabularx}

%----------------------------------------------------------------------------------------
%	SKILLS
%----------------------------------------------------------------------------------------
\vspace{-0.1cm}
\section{Skills}

\begin{minipage}[t]{\linewidth}
    \begin{itemize}[nosep, leftmargin=2em, itemsep=3pt]
        \item \textbf{Programming languages}: C, C++, Python, System Verilog, Regex, Makefiles, ARM, RISC-V, MIPS assembly
        \item \textbf{Tools}: Gem5, Cadence Virtuoso, Eagle CAD, VS code, Vim, Git, Perforce, GDB
        \item \textbf{Parallel Comp Arch}: MOESI, MSIF, Update based protocols (Firefly, Dragon), Memory Consistency Models
        \item \textbf{SoC Arch}: Distributed Shared Memory, Interconnect Networks \& Topologies, CUMA, NUMA, COMA Arch
        \item \textbf{Parallel Programming}: CUDA, MPI, Pthreads
        
        % \item \textbf{FPGA tools}: Altera Cyclone (Intel Quartus prime sw), Xilinx KRIA (Vivado sw), experience in Emulation Build flow
        % \item Proficient in Automotive tools: Canoe, CAPL scripting, DaVinci Developer and Integrator 
        % \item \textbf{Communication protocols:} AMBA AXI, APB, AHB, Avalon MM interface, CAN, UART, I2C
    \end{itemize}
\end{minipage} 

% ---------------------------------------------------------------
%Experience
\vspace{-0.1cm}
\section{Work Experience}
\begin{tabularx}{\linewidth}{ @{}l r@{} }
    \textbf{AMD} (Memory Subsystem Engineer) Santa Clara, CA & \hfill {\small June 2024 - Present} \\[2.75pt]
    \multicolumn{2}{@{}X@{}}
    {
        \begin{minipage}[t]{\linewidth}
            \begin{itemize}[nosep,after=\strut, leftmargin=2em, itemsep=3pt]
                \item RTL performance analysis \& microbenchmarking for Infinity Fabric composed of Interconnect NW, Memory Controller.
                \item Theoretical analysis of mem system optimizations, develop-maintain IP debug tools for performance bottleneck debug.
                \item Worked on AMD's cutting edge products like EPYC, MI400's memory subsystem IP's driving innovation.
            \end{itemize}
        \end{minipage}
    }
\end{tabularx}
\vspace{0.1cm}

\begin{tabularx}{\linewidth}{ @{}l r@{} }
    \textbf{AMD} (Intern) Boxborough, MA & \hfill {\small May 2023 - Aug 2023} \\[2.75pt]
    \multicolumn{2}{@{}X@{}}
    {
        \begin{minipage}[t]{\linewidth}
            \begin{itemize}[nosep,after=\strut, leftmargin=2em, itemsep=3pt]
                \item Designed singlecycle RISC processor with xDRAM integration in SystemVerilog for streamlined testing of emulators.
                \item Created a Python-based Build Differentiator to efficiently compare emulation builds and diagnose failures.
                \item Tool performs a precise comparison of \textbf{70k} build files, identifying version and configuration discrepancies.
                % \item Worked on \textbf{Protium Emulator}, debugging integration and jenkins configuration for regression build.
                % \item Contributed to debugging integration for X2 emulator, build optimization, jenkins configuration for regression build.
            \end{itemize}
        \end{minipage}
    }
\end{tabularx}
\vspace{0.1cm}

\begin{tabularx}{\linewidth}{ @{}l r@{} }
    \textbf{Purdue University} (Graduate Teaching Assistant) West Lafayette, IN & \hfill {\small Aug 2022 - May 2024} \\[2.75pt]
    \multicolumn{2}{@{}X@{}}
    {
        \begin{minipage}[t]{\linewidth}
            \begin{itemize}[nosep,after=\strut, leftmargin=2em, itemsep=3pt]
                \item Taught Code Quality, Test-Driven Development, GDB , Valgrind, Vim for \textbf{Advanced C Programming} class.
                % \item Supported students develop embedded applications on esp32 using micropython in \textbf{Software for Embedded Systems}.
                \item Instructed students on designing multicore, L1 split cache coherent CPU in RTL for \textbf{Computer Architecture} class.     
            \end{itemize}
        \end{minipage}
    }  
\end{tabularx}
\vspace{0.1cm}

\begin{tabularx}{\linewidth}{ @{}l r@{} }
    \textbf{Volvo Trucks} (Embedded Application Engineer) Banglore, India & \hfill {\small Sept 2020 - Jul 202}2 \\[2.75pt]
    \multicolumn{2}{@{}X@{}}
    {
        \begin{minipage}[t]{\linewidth}
            \begin{itemize}[nosep,after=\strut, leftmargin=2em, itemsep=3pt]
                \item Worked on verifying and developing software components in C++ using AUTOSAR architecture. % , as part of  Volvo-Eicher Truck Project.
                \item Developed a testing application using python scripting and Digital IO box hardware prototype to test IC.
                % \item This was a cost-efficient alternative to PLC to automate Instrument Cluster testing using image processing and CAPL. 
                % \item Worked as V3 Verification Engineer for IC 1.0, as part of Volvo-Eicher Truck Project.
                % \item Collaborated on projects to develop Software Components for Instrument clusters in AUTOSAR architecture.
                % \item Developed a Testing application in python and Digital IO box hardware prototype (cost-efficient alternative to PLC) to automate IC testing using image processing and CAPL.  
            \end{itemize}
        \end{minipage}
    }
\end{tabularx}

%-----------------------------------------------------------------------------
%Projects
\vspace{-0.1cm}
\section{Projects}

\begin{tabularx}{\linewidth}{ @{}l r@{} }
    \textbf{Speculative Coherency Modelling in Distributed Shared Memory} & \hfill {\small Jan 2024 - Present} \\[2.75pt]
    \multicolumn{2}{@{}X@{}}
    {
        \begin{minipage}[t]{\linewidth}
            \begin{itemize}[nosep,after=\strut, leftmargin=2em, itemsep=3pt]
                \item Performance modelling of \textbf{Dynamic Self Invalidation} and \textbf{Last Touch Prediction} protocols in cache architecture.
                \item Implemented with DSM \textbf{MOESI directory protocol} for scalable nodes in Gem5 using ruby framework in C++.
                \item Benchmarked Splash3 applications and achieved an average \textbf{speedup of 28\% in DSI and 32\% in LTP}.
            \end{itemize}
        \end{minipage}
    }  
\end{tabularx}
\vspace{0.1cm}

% \begin{tabularx}{\linewidth}{ @{}l r@{} }
%     \textbf{AlexNET - Convolutional Neural Network implementation } & \hfill {\small Jan 2024 - Present} \\[2.75pt]
%     \multicolumn{2}{@{}X@{}}
%     {
%         \begin{minipage}[t]{\linewidth}
%             \begin{itemize}[nosep,after=\strut, leftmargin=2em, itemsep=3pt]
%                 \item Acceleration of convolution, gemm and max pool operations with optimization for run time in CUDA.
%             \end{itemize}
%         \end{minipage}
%     }  
% \end{tabularx}
% \vspace{0.1cm}

\begin{tabularx}{\linewidth}{ @{}l r@{} }
    \textbf{Load Value Predictor Performance Modelling} (Alpha 21264) \href{https://github.com/sharathat45/load_value_predictor/tree/lvp_v1}{[\underline{git}]} 
    & \hfill {\small Aug 2023 - Dec 2023} \\[2.75pt]
    \multicolumn{2}{@{}X@{}}
    {
        \begin{minipage}[t]{\linewidth}
            \begin{itemize}[nosep, after=\strut, leftmargin=2em, itemsep=3pt]
                \item Improving data access time by exploiting data value locality in cycle-accurate microarchitectural \textbf{Gem5 simulation}. 
                \item Implemented Load Value Prediction Table, Load Classification Table, Constant Verification Unit in C++. 
                % \item Benchmarked performance of the processor for SPEC 2006, bzip2, gcc, sjeng, leslie3d, namd benchmarks.
                \item Achieved \textbf{25\%} speedup in \textbf{bzip} and \textbf{15\%} speedup in \textbf{gcc} benchmarks for OoO CPU compared to baseline O3 model.
            \end{itemize}
        \end{minipage}
    } 
\end{tabularx}
\vspace{0.1cm}

\begin{tabularx}{\linewidth}{ @{}l r@{} }
    \textbf{Vortex GPU Coupled Scalar Core Design} \href{https://gitfront.io/r/Sharath/tkpYU7EunMqf/GPU-scalar-riscV-design/}{[\underline{git}]} \href{https://github.com/TehkCode/Vortex-Purdue/tree/ArchV1_3}{[\underline{git}]} & \hfill {\small Aug 2023 - Dec 2023} \\[2.75pt]
    \multicolumn{2}{@{}X@{}}
    {
        \begin{minipage}[t]{\linewidth}
            \begin{itemize}[nosep,after=\strut, leftmargin=2em, itemsep=3pt]
                \item Integrating latency-sensitive RiscV scalar core into the Vortex GPU core, with shared L1 cache in System Verilog.
                \item Implemented high priority kernel scheduler for assigning priority tasks to custom scalar processor for better performance.
                \item Tightly coupled Vortex GPGPU core with scalar core for achieving speedup of \textbf{1.3x} on control flow divergence tasks. 
            \end{itemize}
        \end{minipage}
    } 
\end{tabularx}
\vspace{0.1cm}

% \begin{tabularx}{\linewidth}{ @{}l r@{} }
%     \textbf{Multi Core Processor design} \href{https://gitfront.io/r/Sharath/YQcLhKyPxPYD/CPU-desgin/}{[\underline{git}]} & \hfill {\small Jan 2023 - May 2023} \\[2.75pt]
%     \multicolumn{2}{@{}X@{}}
%     {
%         \begin{minipage}[t]{\linewidth}
%             \begin{itemize}[nosep, after=\strut, leftmargin=2em, itemsep=3pt]
%                 \item Constructed multicore, split cache, pipelined MIPS processor with cache coherence in System Verilog.
%                 \item Incorporated MSI protocol, Branch Predictor with BTB and achieved speed of around \textbf{60 MHz} after Synthesis.
%                 \item Designed cache coherency, bus controller \& Read-Modify-Write functionality for multithreading.
%                 % \item Validated functionality of processor through assembly program simulations in QuestaSim and on FPGA.
%                 % \item Built from scratch, starting with ALU using interfaces, testbench, packages, and functional units. 
%             \end{itemize}
%         \end{minipage}
%     }  
% \end{tabularx}
% \vspace{0.1cm}

\begin{tabularx}{\linewidth}{ @{}l r@{} }
    \textbf{Hardware Acceleration of Neural Network Inference} \href{https://github.com/sharathat45/Vector-dot-product}{[\underline{git}]} & \hfill {\small Aug 2022 - Dec 2022} \\[2.75pt]
    \multicolumn{2}{@{}X@{}}
    {
        \begin{minipage}[t]{\linewidth}
            \begin{itemize}[nosep,after=\strut, leftmargin=2em, itemsep=3pt]
                \item Classified image numerical data from MNIST dataset using KANN API, a lightweight neural network library.
                \item Enhanced sw performance through optimized vector dot product implementation and custom instructions in hardware.
                \item Interfaced DMA/burst transfers with pipelined computation, resulting in \textbf{6.4x(CNN)} and \textbf{16.86x(MLP)} speedup.
                % \item Optimized software performance through profiling and accelerating vector dot product implementation.
                % \item Deployed Combinational, Multi-Cycle custom instructions and Avalon MM Interface Accelerator with Nios 2f.  
                % \item Profiled the software and identified tasks suitable for acceleration. As a result, implemented accelerated vector dot product using floating point addition, subtraction, and multiplication.
                % \item Utilised Combinational, Multi-Cycle custom instructions, Avalon MM Interface Accelerator with Nios 2f  
            \end{itemize}
        \end{minipage}
    }
\end{tabularx}
\vspace{0.1cm}


% \begin{tabularx}{\linewidth}{ @{}l r@{} }
%     \textbf{STRAM design} & \hfill {\small Aug 2022 - Dec 2022} \\[2.75pt]
%     \multicolumn{2}{@{}X@{}}
%     {
%         \begin{minipage}[t]{\linewidth}
%             \begin{itemize}[nosep,after=\strut, leftmargin=2em, itemsep=3pt]
%                 \item Implemented 8T SRAM Array Design layout\textbf{(128 words of 16 Bit length)} for data access in Cadence Virtuoso. 
%                 \item Designed circuits with optimal transistor sizes for equal rise/fall times using Cadence Virtuoso for 45nm process node.
%                 % \item Obtained a sampling rate of \textbf{1000MHz} for the Wallace Tree Multiplier silicon layout (4 Bit). 
%                 % \item Implemented 8T SRAM Array Design layout\textbf{(128 words of 16 Bit length)} for data storage and retrieval. 
%                 % \item Incorporated Write-bitline driver, sense amplifier, and pre-charge driver supporting NOR computation between words. 
%                 % \item Incorporated Write-bitline driver with a 16-bit data-in port, a sense amplifier with a 16-bit data-out port, and a pre-charge driver with a 1-bit input. 
%             \end{itemize}
%         \end{minipage}
%     }  
% \end{tabularx}

%----------------------------------------------------------------------------------------
%  \vspace{-\baselineskip} % Not sure why this extra space is occuring

% \begin{tabularx}{\linewidth}{ @{}l r@{} }
%   \textbf{Air Quality Monitoring System Prototype} (collaboration with NexGen Earth Labs)\\[2.75pt]
%   \multicolumn{2}{@{}X@{}}
%   {
%       \begin{minipage}[t]{\linewidth}
%           \begin{itemize}[nosep,after=\strut, leftmargin=2em, itemsep=3pt]
%                \item Developed the PCB system and prototype using an Esp8266 controller with wifi capabilities. 
%                \item The system includes UART-based sensors and debuggers, SPI-based RTC and SD card storage, and an I2C-based LCD.
%           \end{itemize}
%       \end{minipage}
%   }  
% \end{tabularx}

% \begin{tabularx}{\linewidth}{ @{}l r@{} }
%   \textbf{Autonomous Drone Navigation using VINS (collaboration with UNICITA company)} \\[3.75pt]
%   \multicolumn{2}{@{}X@{}}
%   {
%       The drone navigates using user inputs (from Qgroundcontrol interface) and onboard system inputs from the Visual Inertial Navigation algorithm running on Jetson Nano. HITL(Hardware In The Loop) simulations were done on PX4 using a ROS and gazebo as interactive environments.
%   }  
% \end{tabularx}


%----------------------------------------------------------------------------------------
%	PUBLICATIONS
%----------------------------------------------------------------------------------------
% \section{Publications}
% \begin{refsection}[citations.bib]
% \nocite{*}
% \printbibliography[heading=none]
% \end{refsection}

% \vfill
% \center{\footnotesize Last updated: \today}

\end{document}
