%----------------------------------------------------------------------------------------
%	DOCUMENT DEFINITION
%----------------------------------------------------------------------------------------

% article class because we want to fully customize the page and not use a cv template
\documentclass[a4paper,10pt]{article}

%----------------------------------------------------------------------------------------
%	FONT
%----------------------------------------------------------------------------------------

% % fontspec allows you to use TTF/OTF fonts directly
% \usepackage{fontspec}
% \defaultfontfeatures{Ligatures=TeX}

% % modified for ShareLaTeX use
% \setmainfont[
% SmallCapsFont = Fontin-SmallCaps.otf,
% BoldFont = Fontin-Bold.otf,
% ItalicFont = Fontin-Italic.otf
% ]
% {Fontin.otf}

%----------------------------------------------------------------------------------------
%	PACKAGES
%----------------------------------------------------------------------------------------
\usepackage{url}
\usepackage{parskip} 	

%other packages for formatting
\RequirePackage{color}
\RequirePackage{graphicx}
\usepackage[usenames,dvipsnames]{xcolor}
\usepackage[scale=0.9, bottom=0mm, top =1.2cm]{geometry}

% \usepackage[scale=0.9, bottom=0mm, top =1.2cm, showframe]{geometry}


%tabularx environment
\usepackage{tabularx}

%for lists within experience section
\usepackage{enumitem}

% centered version of 'X' col. type
\newcolumntype{C}{>{\centering\arraybackslash}X} 

%to prevent spillover of tabular into next pages
\usepackage{supertabular}
\usepackage{tabularx}
\newlength{\fullcollw}
\setlength{\fullcollw}{0.47\textwidth}

%custom \section
\usepackage{titlesec}				
\usepackage{multicol}
\usepackage{multirow}

%CV Sections inspired by: 
%http://stefano.italians.nl/archives/26
\titleformat{\section}{\Large\raggedright}{}{0em}{}[\titlerule]
\titlespacing{\section}{0pt}{10pt}{10pt}

%for publicationshttps://www.overleaf.com/project/6390f017c5f74f795fdfe0e5
% \usepackage[style=authoryear,sorting=ynt, maxbibnames=2]{biblatex}

%Setup hyperref package, and colours for links
\usepackage[unicode, draft=false]{hyperref}
\definecolor{linkcolour}{rgb}{0,0,0}  %{0,0.2,0.6} for blue
\hypersetup{colorlinks,breaklinks,urlcolor=linkcolour,linkcolor=linkcolour}
% \addbibresource{citations.bib}
% \setlength\bibitemsep{1em}

%for social icons
\usepackage{fontawesome5}

%debug page outer frames
%\usepackage{showframe}

%----------------------------------------------------------------------------------------
%	BEGIN DOCUMENT
%----------------------------------------------------------------------------------------
\begin{document}

% non-numbered pages
\pagestyle{empty} 
%----------------------------------------------------------------------------------------
%	TITLE
%----------------------------------------------------------------------------------------

% \begin{tabularx}{\linewidth}{ @{}X X@{} }
% \huge{Your Name}\vspace{2pt} & \hfill \emoji{incoming-envelope} email@email.com \\
% \raisebox{-0.05\height}\faGithub\ username \ | 
% \raisebox{-0.00\height}\faLinkedin\ username \ | \ \raisebox{-0.05\height}\faGlobe \ mysite.com  & \hfill \emoji{calling} number
% \end{tabularx}

\begin{tabularx}{\linewidth}{@{} C @{}}
\Huge\textrm{Sharath Hebbur Shivakumar} \\[7.5pt]
\href{https://github.com/sharathat45}{\raisebox{-0.05\height}\faGithub\ sharathat45} \ $|$ \ 
\href{https://www.linkedin.com/in/sharath-hebbur-shivakumar-700879227/}{\raisebox{-0.05\height}\faLinkedin\ Sharath Hebbur Shivakumar} \ $|$ \ 
% \href{https://mysite.com}{\raisebox{-0.05\height}\faGlobe \ mysite.com} \ $|$ \ 
\href{mailto:shebburs@purdue.edu}{\raisebox{-0.05\height}\faEnvelope \ shebburs@purdue.edu} \ $|$ \ 
\href{tel:+000000000000}{\raisebox{-0.05\height}\faMobile \ +1 765-767-3705} \\
\end{tabularx}

%----------------------------------------------------------------------------------------
% EXPERIENCE SECTIONS
%----------------------------------------------------------------------------------------

%Interests/ Keywords/ Summary
% \section{Objective}
% % \vspace{-0.2cm}
% Seeking for Spring Coop and Full time opportunities 

%----------------------------------------------------------------------------------------
%	EDUCATION
%----------------------------------------------------------------------------------------
\section{Education}
% \vspace{-0.2cm}

\begin{tabularx}{\linewidth}{ @{}l X@{} }
\textbf{Master's in Electrical and Computer Engineering at Purdue University}, USA & \hfill Aug 2022 - present \\[2.75pt]
% Specialization in VLSI and Circuit Design &  \hfill GPA: 3.77/4.0\\
\textbf{Courses}: MOS VLSI Design, SoC Design, GPU, Computer Architecture and Design &  \hfill GPA: 3.77/4.0\\
ASIC Design, Digital Design \end{tabularx}

\begin{tabularx}{\linewidth}{ @{}l X@{} }
\textbf{Bachelor's Degree at PES University}, India & \hfill Aug 2016 - Sept 2020 \\[2.75pt]
Electrical and Electronics Engineering, Graduated with Medal, Honors and Rank & \hfill GPA: 9.07/10.0
\end{tabularx}

%----------------------------------------------------------------------------------------
%	SKILLS
%----------------------------------------------------------------------------------------
\section{Skills}
% \vspace{-0.2cm}
\begin{minipage}[t]{\linewidth}
    \begin{itemize}[nosep, leftmargin=2em, itemsep=3pt]
    
    \item \textbf{Programming languages:} C, C++, Python and Bash scripting, Regex, Makefiles, SystemVerilog, MIPS assembly

    \item \textbf{Software:} Gem5, Cadence Virtuoso, Eagle CAD, MATLAB, Labview, VS code, Vim, Git, Perforce, GDB
    
    \item \textbf{FPGA tools}: Altera Cyclone (Intel Quartus prime sw), Xilinx KRIA (Vivado sw), experience in Emulation Build flow
    
    % \item Proficient in Automotive tools: Canoe, CAPL scripting, DaVinci Developer and Integrator 
    % \item \textbf{Communication protocols:} AMBA AXI, APB, AHB, Avalon MM interface, CAN, UART, I2C
    \end{itemize}
\end{minipage} 
% ---------------------------------------------------------------
%Experience
% \vspace{-0.2cm}
\section{Work Experience}
% \vspace{-0.2cm}
\begin{tabularx}{\linewidth}{ @{}l r@{} }
\textbf{AMD} (Virtual Bring Up Intern) Boxborough, MA & \hfill May 2023 - Aug 2023 \\[2.75pt]
\multicolumn{2}{@{}X@{}}{
\begin{minipage}[t]{\linewidth}
    \begin{itemize}[nosep,after=\strut, leftmargin=2em, itemsep=3pt]

\item Created a Python-based Build Differentiator to efficiently compare emulation builds and diagnose failures.
\item Tool performs a precise comparison of \textbf{70k} build files, identifying version and configuration discrepancies.
\item Designed singlecycle RISC processor with xDRAM integration in SystemVerilog for streamlined testing of emulators.
\item Contributed to debugging integration for X2 emulator, build optimization, jenkins configuration for regression build.

    \end{itemize}
    \end{minipage}
}  \\
\end{tabularx}
\begin{tabularx}{\linewidth}{ @{}l r@{} }
\textbf{Purdue University} (Teaching Assistant) West Lafayette, IN & \hfill Aug 2022 - Dec 2023 \\[2.75pt]
\multicolumn{2}{@{}X@{}}{
\begin{minipage}[t]{\linewidth}
    \begin{itemize}[nosep,after=\strut, leftmargin=2em, itemsep=3pt]
        \item Taught Code Quality, Test-Driven Development, GDB , Valgrind, Vim for \textbf{Advanced C Programming} class.
        \item Helped students develop embedded applications on esp32 using micropython in \textbf{Software for Embedded Systems}.
        \item Instructed students on designing multicore L2 split cache coherent CPU in \textbf{Computer Design} lab.
        
    \end{itemize}
    \end{minipage}
}  \\
\end{tabularx}
\begin{tabularx}{\linewidth}{ @{}l r@{} }
\textbf{Volvo Trucks} (Embedded Application Engineer) Banglore, India & \hfill Sept 2020 - Jul 2022 \\[2.75pt]
\multicolumn{2}{@{}X@{}}{
\begin{minipage}[t]{\linewidth}
    \begin{itemize}[nosep,after=\strut, leftmargin=2em, itemsep=3pt]
    
\item Worked on verifying and developing software components using AUTOSAR architecture.
% , as part of  Volvo-Eicher Truck Project.
  
\item Developed a testing application using python scripting and Digital IO box hardware prototype to test IC.

% \item This was a cost-efficient alternative to PLC to automate Instrument Cluster testing using image processing and CAPL. 

% \item Worked as V3 Verification Engineer for IC 1.0, as part of Volvo-Eicher Truck Project.
  
% \item Collaborated on projects to develop Software Components for Instrument clusters in AUTOSAR architecture.

% \item Developed a Testing application in python and Digital IO box hardware prototype (cost-efficient alternative to PLC) to
% automate IC testing using image processing and CAPL.  

    \end{itemize}
    \end{minipage}
}
\end{tabularx}

%-----------------------------------------------------------------------------
\vspace{-0.5cm}
%Projects
\section{Projects}
% \vspace{-0.1cm}
\begin{tabularx}{\linewidth}{ @{}l r@{} }
\textbf{GPU Coupled Scalar Core Design} \\[2.75pt]
\multicolumn{2}{@{}X@{}}
{
\begin{minipage}[t]{\linewidth}
    \begin{itemize}[nosep,after=\strut, leftmargin=2em, itemsep=3pt]
    \item Integrating latency-sensitive scalar core into the GPU core, with shared data scratch pad. 
    \item Implementing thread arbitration for higher-priority warps, to scalar core's instruction scratch pad.
    \item Coupling Vortex GPU core with a RISC-V pipelined core for parallel processing capabilities. 

    \end{itemize}
\end{minipage}
} 
\end{tabularx}
\begin{tabularx}{\linewidth}{ @{}l r@{} }
\textbf{Multi Core Processor design} \\[2.75pt]
\multicolumn{2}{@{}X@{}}
{
\begin{minipage}[t]{\linewidth}
    \begin{itemize}[nosep,after=\strut, leftmargin=2em, itemsep=3pt]
    \item Constructed multicore pipelined processor with cache coherence, adhering to MIPS architecture standards.
    
    \item Incorporated MSI protocol, Branch Predictor with BTB and achieved speed of around \textbf{60 MHz} after Synthesis.

    \item Designed split cache with cache coherence, bus control \& Read-Modify-Write functionality for multithreading.
    
    % \item Validated functionality of processor through assembly program simulations in QuestaSim and on FPGA.
    % \item Built from scratch, starting with ALU in SystemVerilog using interfaces, testbench, packages, and functional units. 
    
    \end{itemize}
\end{minipage}
}  \vspace{-0.1cm}
\end{tabularx}
\begin{tabularx}{\linewidth}{ @{}l r@{} }
\textbf{Modelling Piecewise-Linear Branch Predictor} \\[2.75pt]
\multicolumn{2}{@{}X@{}}
{
\begin{minipage}[t]{\linewidth}
    \begin{itemize}[nosep,after=\strut, leftmargin=2em, itemsep=3pt]
    \item Improving prediction accuracy by learning the behavior of certain linearly Inseparable branches.  
    \item Performance evaluation by simulating predictor with a CPU capable of dynamic execution in Gem5 simulator.
    
    \end{itemize}
\end{minipage}
} 
\end{tabularx}
\begin{tabularx}{\linewidth}{ @{}l r@{} }
{\textbf{Hardware Acceleration of Neural Network Inference} \href{https://github.com/sharathat45/Vector-dot-product}{[\underline{git link}]}}\\[2.75pt]
\multicolumn{2}{@{}X@{}}
{
\begin{minipage}[t]{\linewidth}
    \begin{itemize}[nosep,after=\strut, leftmargin=2em, itemsep=3pt]
    \item Classified image numerical data from MNIST dataset using KANN API, a lightweight neural network library.

    \item Enhanced sw performance through optimized vector dot product implementation and custom instructions in hardware.
    
    % \item Optimized software performance through profiling and accelerating vector dot product implementation.
    
    % \item Deployed Combinational, Multi-Cycle custom instructions and Avalon MM Interface Accelerator with Nios 2f.  

    \item Interfaced DMA/burst transfers with pipelined computation, resulting in a \textbf{6.4x(CNN)} and \textbf{16.86x(MLP)} speedup.
% \item Profiled the software and identified tasks suitable for acceleration. As a result, implemented accelerated vector dot product using floating point addition, subtraction, and multiplication.
% Utilised Combinational, Multi-Cycle custom instructions, Avalon MM Interface Accelerator with Nios 2f  
    \end{itemize}
\end{minipage}
} %\vspace{0.1cm}
\end{tabularx}
\begin{tabularx}{\linewidth}{ @{}l r@{} }
\textbf{STRAM design} \\[2.75pt]
\multicolumn{2}{@{}X@{}}
{
\begin{minipage}[t]{\linewidth}
    \begin{itemize}[nosep,after=\strut, leftmargin=2em, itemsep=3pt]
    % \item Designed circuits with optimal transistor sizes for equal rise/fall times using Cadence Virtuoso for 45nm process node.
   
    % \item Obtained a sampling rate of \textbf{1000MHz} for the Wallace Tree Multiplier silicon layout (4 Bit). 
    
    % \item Implemented 8T SRAM Array Design layout\textbf{(128 words of 16 Bit length)} for data storage and retrieval. 
    
    \item Implemented 8T SRAM Array Design layout\textbf{(128 words of 16 Bit length)} for data access in Cadence Virtuoso. 
    
    % \item Incorporated Write-bitline driver, sense amplifier, and pre-charge driver supporting NOR computation between words. 
    % \item Incorporated Write-bitline driver with a 16-bit data-in port, a sense amplifier with a 16-bit data-out port, and a pre-charge driver with a 1-bit input. 
    \end{itemize}
\end{minipage}
}  %\vspace{0.1cm}
\end{tabularx}

% \begin{tabularx}{\linewidth}{ @{}l r@{} }
% \textbf{Air Quality Monitoring System Prototype} (collaboration with NexGen Earth Labs)\\[2.75pt]
% \multicolumn{2}{@{}X@{}}{
% \begin{minipage}[t]{\linewidth}
%     \begin{itemize}[nosep,after=\strut, leftmargin=2em, itemsep=3pt]
%     \item Developed the PCB system and prototype using an Esp8266 controller with wifi capabilities. 
%     \item The system includes UART-based sensors and debuggers, SPI-based RTC and SD card storage, and an I2C-based LCD.
%     \end{itemize}
% \end{minipage}
% }  
% \end{tabularx}

% \begin{tabularx}{\linewidth}{ @{}l r@{} }
% \textbf{Autonomous Drone Navigation using VINS (collaboration with UNICITA company)} \\[3.75pt]
% \multicolumn{2}{@{}X@{}}{The drone navigates using user inputs (from Qgroundcontrol interface)  and onboard system inputs from the Visual Inertial Navigation algorithm running on Jetson Nano. HITL(Hardware In The Loop) simulations were done on PX4 using a ROS and gazebo as interactive environments.
% }  
% \vspace{-0.05cm} %This will remove 0.5cm of space between the introduction and the next section
% \end{tabularx}


%----------------------------------------------------------------------------------------
%	PUBLICATIONS
%----------------------------------------------------------------------------------------
% \section{Publications}
% \begin{refsection}[citations.bib]
% \nocite{*}
% \printbibliography[heading=none]
% \end{refsection}

% \vfill
% \center{\footnotesize Last updated: \today}

\end{document}
